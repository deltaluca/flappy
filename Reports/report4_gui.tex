\section{Features}

\begin{itemize}

\item The GUI is fully DAIDE compliant and able to observe a Diplomacy game
      played on any DAIDE compliant server.

\item Unit locations, supply centre ownerships and moves are all displayed
      to the user on a scrolling/zoomable map with the whole GUI
      resizable to any reasonable dimension.

\item Game view is controlled by playback buttons with the ability to step
      through turns individually or view the game at a controllable
      pace. These are placed in the title bar together with the connection
      interface. \\[0.5cm]

\begin{figure}

\includegraphics[scale=0.5]{./screenshots/Titlebar.png} \\[0.5cm]

This titlebar also displays the current turn / phase of the game being
viewed.

\end{figure}

\item The GUI includes a command line interface which can be overlayed on
      the map providing resources for debugging moves and inputting
      moves manually as a human-player (Ontop of which like the GUI
      itself, a human-player GUI can be implemented).

\item The GUI is capable of playing/observing a game on any Map, and is
      preloaded with the required resources for playing the standard
      DAIDE map.

\end{itemize}

% --------------------------------------------------------------------------

\section{GUI design}

\paragraph{Choice of language}
For the GUI the HaXe programming language (together with the NME
library) was chosen so as to be able to build a cross-platform GUI
capable of use on Windows/Mac/Linux desktops as well as Android and
OSX mobile devices without worrying about each individual plaform and
it's capabilities.

\section{File formats}

\subsection{Graphic description -- SVG}

An .svg vector graphic file which provides graphic paths for clickable
regions of the map (Selection of regions is implemented in the GUI but
disabled once it was decided there was insufficient time to implement
a human player GUI). Together with a set of named graphics to encode
the locations that units may occupy and supply centre locations.

SVG is a widely-deployed, royalty-free graphics format with an open
specification based on .xml files and can be created by many graphics
editors.

\begin{figure}

\includegraphics[scale=0.5]{./gui/SVG.png}

Note that colours and shape/size of location markers are
arbitrary. Information to be mapped to the real DAIDE map is supplied
in metadata.

\end{figure}

\subsection{Localization}

A simple text file mapping province names used in the SVG file
metadata, to DAIDE protocol IDs. In the future this could be used to
change the language or play the game in a different scenario than
World war I.

\subsection{Texture data}

A set of graphics to be used with trilinear filtering to provide
crisp display of the map at any zoom level with little
performance loss compared to runtime rendering of a detailed and
complex vector graphic for instance.

\section{Technical Issues}
As with the Haskell server, it was necessary to implement a full
implementation of the DAIDE protocol including
serialisation / deserialisation of DAIDE tokens, and parsing / unparsing
of language constructs. As the GUI was first started with the CLI it
was necessary also to build a lexer for DAIDE tokens from their
textual representations and for debug purposes in communicating with
the server.

For these tasks existing open-source HaXe libraries (hlex,hllr) were
used (And extended to provide more type safety in the parser to help
with a grammar as large as that of DAIDE).

Additionally, there is no existing .svg library for HaXe. A reader for
the subset of the SVG format required for the GUI was thus written,
including the transformation of graphic paths into a more suitable
format for mouse selection.

The NME library itself was also forked onto github for the purposes of
providing an extension to it's features used in the GUI.

\paragraph{Design}
The GUI was seperated into 3 packages seperating concerns of the DAIDE
protocol, SVG and other map data, and the GUI itself.\\[0.5cm]

\begin{figure}
\includegraphics[scale=0.5]{./gui/UML.png}\\[0.5cm]
\end{figure}

\begin{itemize}

\item The GUI package encloses all parts of the real GUI including it's
      interface to the DAIDE socket.
\item The DAIDE package encloses everything to do with lexing / parsing 
      serialisation of the DAIDE messages together with a wrapper around
      an Asynchronous TCP socket handling the low-level DAIDE protocol
      dispatching messages to the GUI interface and receiving any messages
      to be sent to the server.
\item The map package encloses everything to do with reading .svg files
      and other related files for maps, together with a struct for 
      enclosing all map information for rendering and though no unsued,
      a fast aabb tree for quickly selecting which polygons reprsenting
      regions are to be tested in mouse selection.
\item The top-level package includes the very small Main class instantiating
      the GUI, CLI and Terminal and binding them together, the Terminal
      being a simple GUI in its own right providing textual input for the
      CLI used by both the Terminal and the GUI interface.

\end{itemize}

\begin{figure}
\includegraphics[scale=0.5]{./gui/Activity.png}\\[0.5cm]
\end{figure}

The entire GUI operates on 2 threads :

\paragraph{Main thread}
The Main thread which deals with most of the work, specifically all
GUI related work as demanded by the NME library which is not thread
safe.

The DAIDE socket thread deals with receiving messages from the server
and can reply to basic low-level DAIDE protocol, sending any complex
messages to the GUI Interface to be processed and entered into one of
two areas:

\begin{itemize}

\item A queue for any non-turn related messages which are processed
      whenever possible by the Timer Event handler forming part of
      the GUI Interface
\item A queue for completed turn records, with any message forming
      part of an on-going turn being accumulated into a record before
      being pushed onto the queue. The handling of this queue in the
      GUI Interface Timer Event handler deals with being in a paused /
      playing state and any delays necessary to keep the speed set by 
      the UI.

\end{itemize}


