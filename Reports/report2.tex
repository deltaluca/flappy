\documentclass[12pt]{article}
\usepackage{fullpage}
\usepackage{cite}
\usepackage{datetime}
\usepackage{geometry}
\geometry{verbose,lmargin=3cm,rmargin=3cm}

\title{Automated negotiation in the game of diplomacy - Report 2: Progress \& Revisions}
\author{Matthias Hueser, Andras Slemmer, Luca Deltodesco, Luke Tomlin, Cliff Sun}
\date{\today}

\begin{document}
\maketitle

\section{Project progress}

\subsection{Recap: User stories for 2nd iteration (weeks 3 - 4)}

\begin{itemize}

\item Our primary user story for the 2nd iteration was to have the messaging system
implementing the DAIDE interface in place. The interaction between
the GUI client and the server should include messages up to press level
10 and implement those in a stable way. To achieve robustness the server should
recover from mal-formed messages from faulty clients. From a client perspective
the server should always be responsive even though it is not yet advancing the game state in
response to player messages (see second user story for this)

In effect the interface between the HaXe client (produced for demo purposes)
and the Haskell server should be seamless with no loss of connectivity and 
acknowledgement of messages in both directions.

\item The second aim of the iteration was to produce a skeleton for state modification
and implementing in particular the complex unit standoff rules of Diplomacy.

\item Lastly users should be able to play the game with other maps apart the 'Standard' one.
To enable this we needed to define a representation of a 'Diplomacy map'. This includes
a facility to expand the standard maps which is seen as a special case of the
first requirement.

\subsection{Progress on 'Messaging system' and 'Client-server interface'}

\begin{itemize}

\item The messaging component is working and has been validated with user requests
ranging up to press-level 10. With regards to the HaXe client a GUI interface is
seen as a key requirement but was not necessary to implement the Client-Server
protocol. Currently a command-line interface exists which is used to validate
the correctness of communication.
 
\item In particular on the Haskell server side parsing of messages works correctly.
Due to the complexity of the language specification this

\item The HaXe client currently consists of a command-line interface which can accept
messages, parse them and send a response to the server. 

\end{itemize}

\subsection{Progress on 'Game state modification component'}

We could not make as much progress as we liked on the 'state modification' part of
the server. This was largely due to dependencies to the definition of map representation
which was finished only relatively late in the time-box. One possibility was to allow
divergent code for both and later glue it together but we assessed this to be costly and
confusing down the line. So a delay was deemed to be the better alternative.

\subsection{Progress on the 'Map representation'}

\begin{itemize}

\item The representation of the map is complete. It includes primitives to represent 
the topology of provinces and their types. There are large overlaps with the internal
game state representation and this was taken into account to avoid code duplication
in future iterations.

\item The expandable-map feature (or different maps entirely) is implicit
from the way the map is stored and represented and caused no further code
modifications. 

\end{itemize}

\section{Technical issues during the latest iteration}

\subsection{Error recovery in Message passing}

There was a problem implementing the DAIDE low-level error messages.
If a message fails to parse, an error token needs to be inserted at
the erroneous part of the stream. However the parsing tool of our choice - 
the parser combinatory library Parsec - types the input message
as a constant stream and cannot insert new tokens into it.

A solution we came up with is to save the position where the parsing
failed and then insert the error token outside of Parsec before resuming
the parsing at the same position in the input stream

\subsection{Ambiguity of DAIDE language}

To complicate matters certain constructions of the grammar were hard to
parse without substantial look-ahead. Since we could not change the existing
DAIDE language for compatibility reasons we had to extend the code to deal
with these corner cases.

\section{Schedule revisions}

Given that the schedule slipped due to the above issues we had to 
discuss whether it made sense to drop user stories from future iterations. 
In our case this was not necessary because our project aims are generic.
in nature. This means that we aim to create an experiment textbed which we can augment
with multiple bots using different paradigms (for an overview see section
on AI design). This means that we are very flexible as we reach the end of 
the project since we can drop the implementation of a particular AI paradigm
with no impact on the running system. 

\subsection{Requirements for next iteration (weeks 5-6)}
 
While the game framework (Haskell server) settles into place the 
team and the supervisor agreed that it is of utmost importance to
proceed to the 'Artificial intelligence' / 'Negotiation' phase in the
next iteration.

\subsubsection{Towards the Artificial intelligence}

As discussed above all future work seeks to establish an experimental
test-bed to assess different approaches to AI and negotiation.

Accordingly we chose an iterative development approach to deliver
different AI techniques incrementally. Proceeding in this way we will always 
have a fall-back should certain AI paradigm prove technically challenging.

Having said that we require that all bots share a common structure so
we could integrate them after performing a first assessment of 
playing strength. The team agreed that to develop competing bots with
no exchange of ideas will not lead to satisfying project result. 

While this establishes the 'spirit' of next week's development the 
immediate problems we need to tackle are as follows:

\begin{itemize}

\item Formulate a game tree search algorithm.

\item Select a heuristic to score game states. 

\item Create a 'pattern-weight' database as in [SHA08] to enable
our AI player to learn over the course of several games.

\item Explore the design space for 'negotiation algorithms'.

\end{itemize}

\subsubsection{Order resolution}

At the end of a turn all players simultatenously reveal their orders to the 
server. In response the server will have to perform the following actions:

\begin{itemize}
\item Checking that all orders are valid with respect to the syntax. This is
already implemented by the message parser, who will reject invalid messages.

\item When orders conflict, the server will apply a complex set of rules
derived from the 'Diplomacy manual'. This includes correctly determining
the result of 'standoff' situations, where support orders by other units
need to be taken into account.
\end{itemize}

\subsection{Requirements for iteration (weeks 7-8)}

By this time we should have delivered a non-negotiating bot that 
can be evaluated against the existing DumbBot algorithm. Experiments / 
benchmarks will need to be run to determine any weaknesses that need
to adressed in further development.

\section{Division of work in the next iterations}

The division of work outlined in the previous report is adapted as follows: 

\begin{itemize}

\item Luca / Andras perform deep-testing of the server-client interface
to eliminate any remaining bugs which have not been caught yet.

\item Matthias concentrates on exploring the design space for AIs and
will implement logic to resolve the movement orders for a particular 
round of the game.

\item Luke / Cliff start devising data structures needed for an 
automated player, while finalizing the map representation and
refactoring the existing code base.

\end{itemize}

\section{Future project progress measures}

Once some alternative AI algorithms are in place (Week 9-10) they will be matched against
each other on the DAIDE server. Since every bot implements the DAIDE protocol
it will also play against other DAIDE-conforming bots, of which
a large number exists. From this point on the playing strength of our
bots (which we hope to combine into one system ultimately) will be main 
progress measure of the project, besides stability of the server and quality
of user experience in the GUI client.

\section{Design of the AI player(s)}

\section{Challenges for an AI player in 'Diplomacy'}

Diplomacy is generally seen as an interesting sandbox to explore
different AI techniques for multiple-player strategy games.

It can be classified as a multi-player, cooperative, zero-sum game
with imperfect information, all of which have important ramifications
for the development of an AI.

Each player has imperfect information because the game state is not
changed in reaction to each players movement separately. Instead a 
so-called 'move-resolution' phase takes place. Here players formulate
their moves in secret and then submit them simultaneously to the game
server which resolves any conflicts, 'standoffs'
in Diplomacy parlance. Being unaware of the others players actions means
that the immediate effects of its own movement orders are uncertain. The
game mechanics dictate that at the end of a turn a chain of so-called 
stand-offs might occur, in case several move order conflict. As a result
an AI player is uncertain of the effect of its own actions. Another matter
that increases uncertainty are support orders that bias the resolution of
stand-offs in favor of a particular player. 

Diplomacy is cooperative since negotiation is at the heart of the game:
It allows players to communicate their short/long-term intentions before
their actual moves are revealed. For instance in stand-off situations it will
enable two players to attack a field of a third-party. In general, negotiation
will decrease uncertainty but it can be a double-edged sword. Players might
pretend their actual intentions but actually conspire to betray each other. Hence
in essence an AI player will need to establish whom to trust when considering
his options based on the messages of other players.

\section{Iterative AI development}

\subsection{Primitive hold-bot}
A common feature of every bot is the ability to send trivial but DAIDE-
conforming messages to the server. Hence the first milestone will
be a 'Hold-bot' which is able to play a game, but will 'hold' on
every move. This player has extremely weak playing strength but will
not impede the progress of the game. When receiving a negotiation
message it will respond with a default message.

\subsection{Random bot}
The random bot is an evolution of the 'Hold-bot' and can re-use most
of its code-base. The difference is that it is selecting a move
uniformly from all possibilies in the current game-state. As hold-bot it 
uses default responses in negotiation. We expect this player to be measurably
stronger 'Hold-bot' but still fare poorly against existing AI bots.

\subsection{Heuristic search bot}
From here on a layered approach is envisioned where generic techniques
like 'Game tree search' are implemented first. The reason for this is that 
search is generally considered as the least risky to implement. Apart from
a utility function mapping a game state to a certain metric vector no 
domain knowledge of the game is required. It is considered a primitive
approach but is required to create a bottom layer which can be refined
by learning approaches on the top layers.

\subsection{Tactical bot}
This bot will augment the heuristic search with hard-coded tactics and
the use of an opening book. All of this requires detailed analysis of 
playing techniques and creating a database of typical responses. This should
result in average playing-strength but is still primitive as the bot is not
able to learn from previous actions.

\subsection{Learning / Strategy-bot}
As discussed in the next section powerful AI play requires learning of
techniques and responses to typical game situations. A database of 
'pattern-weights' and 'temporal-difference learning' will enable this bot to consider the
long-term effects of its moves and improve with increasing experience. Notice
that at this stage the bot will create knowledge about the 'Game of
Diplomacy', especially when it determines favorable aspects of a game using
the 'pattern-weights'.

\subsection{Negotiating bot}
Once the previous bots are working a negotiation feature can be implemented
as a new component with relatively low risk. This is important because we
want to measure the additional playing strength an AI player gains through
negotiation. The 'Strategy' and 'Negotiation' components should work together as
follows:  as a result of a negotiation session particular collective strategies
is proposed, which is then sanity-checked by the 'Strategy' component to avoid grave mistakes. 
The negotiation bot itself can be broken up into two separate components:

\subsubsection{Alliance / Trust model}
An AI bot needs to be able to reason about its relation-ship with the other
players. For instance it would be foolish to work out elaborate strategies
with a player which has repeatedly stabbed into the back the AI player. 

\subsubsection{Strategy formation}
Based on the proposals / messages of other players a short- or longterm
strategy is formed. This part has rich interaction with the 'Strategy'
planner component of the 'Learning bot' presented above.

\section{Character / personality of an AI player}
This is not really a refinement of a previous bot but can be implemented 
independently. The idea is that algorithms of existing bots are tuned to 
give rise to a particular playing style. For example an offensive player
is more likely to launch reckless attacks on enemy supply centers or stab
an opponent in the back (stabbing is a well-established technique and is
discussed on DiplomacyPouch). On the other hand a defensive player gives
priority to secure the own supply-centers. For an negotiating bot the
character also determines how often other players are betrayed etc.

Throughout the implementation of the two components 'search' and 'knowledge 
database' emphasis is laid on generality which enables a close integration
with the 'negotation module', which enables intelligent forming of 
alliances and communication of intentions to other players.

\subsubsection{AI Bot Requirements}

\begin{itemize}
\item Such as prioritising keeping all your provinces in your territory/one
area
\item Or trying to capture as much opposing provinces as possible and not
care as much about your own
\item Or capturing other provinces may be more important than keeping your
own and building as many supply centres as possible

\end{itemize}
\item Many possibilities here and taking the time to experiment around with
the tactics will show us which ones are more effective against others
\end{itemize}

% Explanation of the Artificial intelligence architecture,
% including pattern-weights, distributed solving and search heuristics

\section{Details of the proposed AI iterations}

\section{Short-term game tactics}

For short-term game tactics a search with minimal look-ahead can be used.

\subsection{Searching the game tree}

Searching a game tree is considered a generic technique for playing 
games that have discrete states connected by edges which represent movement
orders. As outlined in the introductory section there is uncertainty about
the effects of movement orders, making the viability of this approach doubtable. 

Besides at any given state of the game there is large number of moves possible
for each player, giving rise to a large branching factor in the game tree. [SHA08]
have estimated the number of a possible moves in a game situation to be on the order
of $2^{91}$ to get a feeling for the numbers involved. Even moderate look-ahead in
the search space almost certainly puts a strain on the computational resources.
To overcome this difficulty, search heuristics for pruning the game tree and scheduling
the exploration order of the game branches are alternative-less extension of a
brute-force search. Both techniques can be implemented through so-called utility function
defined on a game state. We discuss the structure of utility functions for Diplomacy
in a later section.

As [SHA08] have argued, modelling the game with payoff matrices is similarly mis-guided
as a brute-force game tree search and so we have not considered it for our AI player.

\subsubsection{Search heuristics}

Whenever a brute-force search is not possible we need to assign a utility to an 
intermediate game state, in which the winner has not yet been determined. On the 
one hand this enables us to cut-off a game-tree search at a shallow level and
still propagate move values upwards through a MinMax procedure. On the other hand
they enable us to guide the MinMax by scheduling the exploration of branches or
pruning branches with limited attractiveness altogether. Even generic heuristics
like AlphaBeta-Pruning or NegaScout that do not use any knowledge of the game give
speed improvements of several orders of magnitude [REFERENCE]. Hence we expect
informed searches to give us considerable latitude to search deeper. 

\subsubsection{State evaluation function}

The aforementioned state utility function will need to incorporate knowledge
of the game to be effective. As the objective of the game is to control the 
majority of the supply-centers any utility function should consider how likely
it is to gain additional centers from a given game state. [Loeb92] propose to 
weight the control centers controlled by a player. Supply centers that are
prone to be attacked in the next turns have a smaller weight whereas ones with
a solid defence base have high weight. All of these functions are blind to 
alliances / cooperations between players in the sense that they treat every 
player as an adversary. A better heuristic would consider the positions of
all allied players collectively, given that there exists some trust bond
between the allied players. 


\section{Long-term strategy templates}

Selecting a long-term strategy will either require hard-coded 
knowledge of the 'Diplomacy' domain or a self-learning approach which
bootstraps an experienced AI starting from primitive playing techniques.
Storing and retrieving learned patterns about the game will be key to build
a catalogue of patterns which can be used to select a suitable strategy.

\subsection{Strategy acquisation}

Several choices: 

a) Hard-coding of offensive / defensive 
   strategies that are played in certain
   situations

b) Start out with rudimentary strategies that
   are extended through learning from previous games.
   (see section 'Learning on previous game-databases')

\subsection{Expert game knowledge}

To improve the AIs playing skills over the lifetime, a
knowledge base with various parts needs to be constructed.

According to the current game state and other parameters, such
as the remaining time in the game or a model of the other players
the agent picks a response rule and generates from it a candidate
move. 

As [SHA08] propose, a game database contains partial-state -> response movement
pairs. A partial state is a high-level description of a state, omitting
details which are not relevant to the formulation of strategy / tactics. If
such a mapping is not found the current state is classified and a candidate move
is created using search techniques. Having found this move, a new entry in the 
database is created. At regular time-points in the future a payoff is calculated
(based on future game metrics) and associated with the new partial-state -> response pair.
The motivation behind this is to rank responses to typical game situations. This
approach reduces the computational load as search is a very costly operation in the
game of Diplomacy. Besides ranking newly-created response upon termination of a game
all used movements are re-ranked using a process called 'Temporal difference learning'.

\subsubsection{Partial game-state encoding}

To enable storing the game data-base its representation must be boiled down to 
the basics required for informed decision making. If necessary additional data
structures like graphs need to be constructed to embody higher-order knowledge
not immediately obvious from a primitive description of the map. 

As [SHA08] suggest, we can construct a hierarchy of pattern-weights where more 
general ones 

a) The current position of all players units
b) A model of all other players (see later) in that situation 
b) Game parameters, including game phase
c) The history of the game


\subsubsection{Opening games}

The opening of games is a special case since a large number
of tournament opening books exist. Since these initial states are amenable to
analysis we can assume that master players had ample opportunities to
pick optimal strategies. Our AI agent has access to such a collection
and will pick a suitable move according to his denomination.

\subsection{Move generation}

Given a description of the game state the AI players classifies
it with respect to several properties and assigns to it a 'partial state'
which is taken from an evolving pattern database. For example in a cooperative
setting a coordinated attack on an opponent would classify as a distinct
'partial state'. As this shows the description should be formulated a level
above move descriptions, so that a chain of movement emerges.

\section{Negotiation component}

The sub-system of the AI responsible for negotation
is seen as virtually independent from the search techniques
and expert database discussed above. Nonetheless it is 
considered vital when competing against players which
use negotation. Consider the situation where offensive alliances
are created against the AI player. To overcome this aggression 
the AI player needs communication to organize defense and launch
retaliation attacks. The product of negotiation are commonly
collective strategies which guide the army movements of the
individual players. In order to safe-guard against other
players who propose reckless strategies those are always
cross-checked against the movements favoured by the search
and knowledge base components.

\subsection{Internal model of other players}

\subsubsection{Trust}

Trust as a relationship between two players removes contingency
from the 'move search process'. This is because certain actions such
as an attack on its own units can be ruled out if a substantial bond of trust
has been established. Besides consistent actions, communication
is needed to forge such relationships among players over time. 

\section{Game meta-heuristics}

\subsection{Learning on previous game-databases}
Learning through pattern-weights (see SHA08):
\vspace{-2mm}
\begin{itemize}
\item[] Idea: Adapt generic initial strategy by performing 
self-play, temporal difference method
\end{itemize}
Reinforcement learning:
\vspace{-2mm}
\begin{itemize}
\item[] Idea: Represent the environment and use a stochastic transition model 
contingent on other players moves (could try to fill in gaps of knowledge with 
a characterization of other players)
\end{itemize}

\section{Project management}

\subsection{Software Engineering Process}
\begin{itemize}
\item Currently an agile software development model is still being followed,
specifically the team development still is much like the Scrum model
\item Sprints i.e. 2 week iterations are planned so that at the end of each
sprint a deliverable can be presented to the Product Owner (i.e. our
supervisor)
\item The next iteration is weeks 5-6 followed by the final iteration which
is weeks 7-8. Week 9 will be allocated as a buffer week to complete
anything that was not completed in the final iteration/sprint
\item There is a meeting after each iteration to discuss how the current
iteration/sprint went and if any goals/aims need to be readjusted
for the following iteration
\end{itemize}
\subsubsection{Scrum Meetings}
Scrum meetings are weekly meetings designed to keep group members updated with 
the current course of the project and to highlight any problems and/or 
interesting points of notice. The meetings generally start with a discussion of 
the progress of each individual member's tasks and whether they are on target for 
the end of the iteration. Any problems encountered so far by the team members 
will also be highlighted and targets/deadlines will be adjusted accordingly. 
Team members will also have to mention if they are particularly busy during a 
week so that necessary precautions can be made to shift work to the other 
members in order to maintain the target for the current iteration. 
A logbook is used during the meetings to record anything mentioned 
above and to write down the aims for the next meeting and/or iteration. 
After each meeting, the logbook is updated to the github repository to keep a 
backlog of all meetings and for reference.

\subsection{Team collaboration}

\begin{itemize}
\item Working in a team means that people have different skills and therefore
may end up contributing more than others, but this can be a positive
as the most effective way to work as a team is to use everyone's skills
effectively
\item The two main languages used are Haskell and Haxe. There is one member
who is strongest at Haskell and one who is strongest at Haxe

\begin{itemize}
\item This has meant that for the first two iterations, the two members
mentioned above have led the project and contributed the most
\item However for the next 2 iterations, contributions should be more even
as the rest of the group has had time to learn the languages required
and can make a greater contribution
\end{itemize}
\end{itemize}

\section{Ethical and Environment Impact}

\subsection{Ethical impact}

Users who register for a game on the system may be required to leave
their email addresses and names for authentication. This means their
personal details need to be stored safely and ensure that it is stored
safely and away from any possible attacks on the server. In addition
this personal information needs to be handled with care and that it
is never given out under any circumstances.\\
In order to avoid any ethical issues we would have to ensure that
only team members have access to the database holding this information
and perhaps we would need some sort of encryption on the table itself.

\subsection{Environmental Impact }

A lot of the evaluation work will consist of running experimental
games between the different bots / human players to assess their 
strengths and weaknesses. To conserve computing resources in the 
department we will run the game server and all AI bots on the same
machine concurrently. 

With respect to exploratory reading we discovered that printing
out all background material and having all logbooks on paper is wasteful
and inefficient. On the one hand discussions pertaining to the
whole project are better placed on the central GitHub repository, since 
distribution is straightforward. 

% Need to add some stuff here!

\end{document}
