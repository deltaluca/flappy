% Initial report, skeleton

\documentclass[]{article}
\usepackage{mathptmx}

\begin{document}
\title{Report 1 : Inception}
\author{Luca Deltodesco, Matthias Hueser, \\Andras Slemmer, Cliff Sun, Luke Tomlin}
\date{\today}
\maketitle

\section{Key Requirements}
\textit{"Up to 10 most critical or high-risk, but definitely essential, requirements that your completed system will meet; their realisation should secure at least grade B. "}

\section{Extensions}
\textit{"Additional features or requirements that, if implemented satisfactorily, could contribute to the group getting an A (provided the group scores well on all other criteria). "}

\section{Choice of Development Method}
\textit{"Suggested (blend of) agile software development method you intend to use.
A description of the anticipated methods and practices for code testing and version control.
Identification of needed information technology.\\
Your project management: how you measure project progress or velocity, what roles and responsibilities team members have throughout project, how often and in what form (e.g. Scrum meeting) will you meet, etc. "}

\section{Draft Schedule}
\textit{"A well-considered but non-binding estimate of how the project should progress:\\
    Outline of anticipated number of time-boxed iterations you will use\\
    Tentative list of features or milestones that each iteration should deliver in executable software\\
    For each feature, an analysis of the technical components necessary for their realization.\\
    Identify which components are supposed to work for each iteration."}

\section{Detailed Plan for First Iteration}
\textit{"Specification of length of, and dates for, your first time-boxed iteration of software development.\\
Detailed mapping of tasks for the iteration, wg with a Sprint backlog.\\
Identification of potential risks during first iteration.\\
Planned progress measure for first iteration. "}


\end{document}