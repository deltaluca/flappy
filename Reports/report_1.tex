%% LyX 2.0.0 created this file.  For more info, see http://www.lyx.org/.
%% Do not edit unless you really know what you are doing.
\documentclass[english]{article}
\usepackage{mathptmx}
\renewcommand{\ttdefault}{mathptmx}
\usepackage[latin9]{inputenc}

\makeatletter
%%%%%%%%%%%%%%%%%%%%%%%%%%%%%% User specified LaTeX commands.
% Initial report, skeleton


\usepackage{mathptmx}



\makeatother

\usepackage{babel}
\begin{document}

\title{Report 1 : Inception}


\author{Luca Deltodesco, Matthias Hueser, \\
Andras Slemmer, Cliff Sun, Luke Tomlin}


\date{\today}

\maketitle

\section{Key Requirements}
\begin{enumerate}
\item AI is capable of playing an entire game of Deplomacy from start to
finish.
\item AI is capable of negotiating with other players in the specified negotiation
language.
\item Software is capable of handling and organising a game of Deplomacy
involving multiple players/AIs.
\end{enumerate}

\section{Extensions}
\begin{enumerate}
\item AI is capable of beating other less complicated AI (eg. ones that
do not negotiate, with simple negotiation tactics etc.).
\item Software is extendable to allow other AIs to play (eg. Daide).
\item Software is runnable on multiple platforms.
\item AI is able to negotiate in a meaningful way, based on previous gameplay
within a game.
\end{enumerate}

\section{Choice of Development Method}


\paragraph*{Methodology}


\paragraph*{Code testing}

Unit testing and other code testing using QuickCheck.


\paragraph*{Version control}

Handled by Git, using GitHub as a central repository.


\subsubsection*{Required technology}
\begin{description}
\item [{Server}] Haskell : Concurrent programming, Network Programming
\item [{GUI}] Haxe : GUI creation, cross-platform availability
\item [{AI}] Haskell
\end{description}

\subsubsection*{Project Management}


\paragraph*{Progress}

After the initial server and game creation is completed, progress
on the AI can be made. AI progress should be fairly simple to gauge,
as it gains negotiation functions and performs better against other
opponents (human/AI). \\
Additionally, the visuals and interactivity of the game itself will
become more advanced and user-friendly as the GUI improves. (eg. from
just a simple CLI -\textgreater{} GUI/CLI -\textgreater{} GUI -\textgreater{}
GUI with prettier graphics...)


\paragraph*{Roles and responsibilities}


\paragraph*{Meeting and scheduling arrangements}


\section{Draft Schedule}

\textit{\char`\"{}A well-considered but non-binding estimate of how
the project should progress:}\\
\textit{ Outline of anticipated number of time-boxed iterations you
will use}\\
\textit{ Tentative list of features or milestones that each iteration
should deliver in executable software}\\
\textit{ For each feature, an analysis of the technical components
necessary for their realization.}\\
\textit{ Identify which components are supposed to work for each iteration.\char`\"{}}


\section{Detailed Plan for First Iteration}

\textit{\char`\"{}Specification of length of, and dates for, your
first time-boxed iteration of software development.}\\
\textit{ Detailed mapping of tasks for the iteration, wg with a Sprint
backlog.}\\
\textit{ Identification of potential risks during first iteration.}\\
\textit{ Planned progress measure for first iteration. \char`\"{}}
\end{document}
