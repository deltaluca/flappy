\documentclass[11pt]{article}
\usepackage{fullpage}
\usepackage{cite}
\usepackage{datetime}
\usepackage{geometry}
\usepackage{graphicx}
\geometry{verbose,lmargin=3cm,rmargin=3cm}

\title{Automated negotiation in the game of diplomacy - Report 4: Final Report}
\author{Matthias Hueser, Andras Slemmer, Luca Deltodesco, Luke Tomlin, Cliff Sun}
\date{\today}

\begin{document}
\maketitle

\section{AIs in the Game of Diplomacy, an overview}
\subsubsection{Who this applies to}
This presentation should interest two groups of people - those who play the multiplayer strategy-game Diplomacy, and those who have an interest in machine learning and AI, specifically dealing with negotiation between human and other AI players. Additionally, it may be of interest to anyone with an affinity for functional programming. In this project we dealt with several different facets of AI learning techniques, and approached it in an iterative format, increasing the complexity with each iteration.
\\
The results of this project are two-fold - a functional, flexible framework for creating AI for The Game of Diplomacy, along with an interesting foray into AI developement for multiplayer, round-based strategy games using a variety of technicques and methods, a side-effect of which is the creation of several different AIs that can compete against human and non-human players alike.
\subsubsection{An introduction to Diplomacy and our product}
The Game of Diplomacy is a round-based multiplayer strategy-game set in post WWI
Europe. Each player controls a power which initially operates from a number of 
home-bases. The ultimate objective of each player is to control the majority of
supply-centers on the map. A game is structured into a number of rounds, in which
each player formulates a move order to conquer additional supply centres or
defend existing ones. Besides the necessity to create effective long-term
strategies negotiation is integral to the game-experience. There is a thriving
internet-based community around the Game of Diplomacy and many AI players, so
called bots, have been created in the past. Most of the existing AI players are
quite effective decision-makers in general but few of them support negotiation
or reasoning about the relationship of the powers in the game. Our project aims
to create a framework for Diplomacy, including a GUI client, a game server and a
collection of bots. Crucially the highest evolution of our series
of bots should be able to analyze and act upon messages of human / AI players and
issue messages to other players. Finally experiments are conducted to determine whether there is a reward associated with negotation capabilities. This will require careful study and implementation of automated negotiation techniques from other domains in AI/Game Theory. Thus the project gives us ample opportunity to be
creative and test different approaches using the DAIDE-conforming game framework
also produced for the project. DAIDE is a standard client-server protocol which
serves as a 'common language' for different Diplomacy bot implementations, 
enabling them to play a game together on a conforming server.
\section{The flappyAI framework and individual AIs}
\section{Softare Engineering Issues}
\section{Validation and Conclusions}
\section{Bibliography}
\section{Appendix}


\end{document}
