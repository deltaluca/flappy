%% LyX 2.0.0 created this file.  For more info, see http://www.lyx.org/.
%% Do not edit unless you really know what you are doing.
\documentclass[english]{article}
\usepackage[T1]{fontenc}
\usepackage[latin9]{inputenc}
\usepackage{babel}
\begin{document}

\title{Automated negotiation in the game of diplomacy}


\title{Report 2 : Progress \& Revisions}


\author{Luca Deltodesco, Matthias Hueser, \\
Andras Slemmer, Cliff Sun, Luke Tomlin}


\date{\today}

\maketitle

\section{Progress}


\subsection{Reminder of targets for 2nd iteration (weeks 3 - 4)}

Start initial work on the AI. Continue to improve and develop game
foundation (work will need to be split between team members here).
AI should be able to play a 'holding' game and fulfill all requirements
to play with other AIs. Begin research into tactics and how we will
develop the AI.


\subsection{2nd Iteration (week 3 - 4) Progress}
\begin{itemize}
\item Currently have a working Haskell server which can communicate with
the front-end interface written in Haxe. 
\item Parsing on Haskell server side (using Parsec) is complete up to press
level 10 
\item Front-end interface currently consists of a command-line interface
which can accept messages, parse it and send and recieve the message
to/from the server
\item Representation of the map is complete, i.e. provinces, their adjacencies
and whether they're land/coast/sea
\item Ability to have expandable maps (or different maps entirely) is implicity
from the way the map is stored and represented
\end{itemize}

\subsubsection{Problems in the latest iteration:}
\begin{itemize}
\item There was a problem initially implementing the Daide low-level errors.
If a message fails to parse, an error token needs to be inserted at
the place/token it failed to parse. However Parsec takes the message
as a stream and therefore it is not possible to insert into a stream. 
\item The solution is to save the position where the parsing failed and
then insert the error token outside of Parsec where the message is
a list rather than a stream
\end{itemize}

\section{Revisions}


\subsection{Changes/Revision}


\subsubsection{Time Constraints}
\begin{itemize}
\item Writing the Haskell server and parser has taken longer than expected
and with all the time needed to do courseworks and industrial placement
applications, the project has gone behind schedule
\item This means that there is less time to work on the AI and negotiation
features but after discussion, the team has decided that there is
still sufficient time to complete all current objectives
\end{itemize}

\subsubsection{(AI) Bot Design }
\begin{itemize}
\item Following on, the focus from now until the end of the current iteration
will be on designing and implementing a bot with effective AI. An
iterative approach will be taken for implementing the bot to ensure
that features are properly implemented and that there is always a
basis to work on (see below)
\item Non-negotiating bot can be tested against Dumb-Bot for benchmarking
and to make more optimisations and adjustments to it
\item Final negotiating bot can then be tested against other negotiating
bots to see it how it matches up, this will be the main test to see
the negotiating AI is an improvement over existing AI's
\end{itemize}

\subsubsection{Core/Extension Changes}
\begin{itemize}
\item Currently none of the core functionality has changed and the plan
is still to implement all the features and extensions that were stated
in the first report
\end{itemize}

\subsubsection{Project Management and Software Engineering Practices}
\begin{itemize}
\item Currently an agile software development model is still being followed,
specifically the team development still is much like the Scrum model
\item Sprints i.e. 2 week iterations are planned so that at the end of each
sprint a deliverable can be presented to the Product Owner (i.e. our
supervisor)
\item The next iteration is weeks 5-6 followed by the final iteration which
is weeks 7-8. Week 9 will be allocated as a buffer week to complete
anything that was not completed in the final iteration/sprint
\item There are multiple meetings during a week which enables the team to
get together to discuss current progress and any problems encountered
so far
\item There is also a meeting after each iteration to discuss how the current
iteration/sprint went and if any goals/aims need to be readjusted
for the following iteration
\end{itemize}

\subsection{Revised Schedule for next iteration (weeks 5-6)}

Targets for the next iteration were discussed, more specifically on
how to design the AI. The team further discussed how to split up the
work from this point onwards and decided that 2 people would work
on creating the initial AI bot and the other 3 would work on finishing
the server (i.e. order resolution). The following is a more detailed
account of what the deliverable should look like at the end of the
next iteration/sprint:


\subsubsection{Order resolution implementation}

At the end will be able to resolves all orders in one round sent by
the players
\begin{itemize}
\item Checks that orders are tactically valid and calculate the end result
of all orders, i.e. map state after the round

\begin{itemize}
\item Checks if orders clash or conflict and resolve them correctly
\end{itemize}
\end{itemize}

\subsubsection{AI Bot Requirements}

The following points show an iterative and high-level approach that
is to be taken in building up a bot which houses the AI features (further
discussed below in {[}2.3 AI Techniques and Design{]}:
\begin{enumerate}
\item Initially create a Hold-Bot which will hold for every move

\begin{itemize}
\item Hold-Bot will essentially serve as a template to build upon and will
need to be able to respond to messages from the server (such as negotiation
requests) with a default response (as such features have not been
implemented in the bot at this point)
\end{itemize}
\item Features can then be added to Hold-Bot to create Random-Bot

\begin{itemize}
\item Random-Bot will generate a list of moves that it can make (based on
the current state of the game and map) and will randomly choose a
move

\begin{itemize}
\item AI techniques for being able to search through the game state to generate
a list of possible moves are discussed in {[}2.3 AI Techniques and
Design{]}
\end{itemize}
\item Random-Bot will still respond to messages from the server in a default
way (like Hold-Bot does) but when asked to make a move will choose
a random one
\end{itemize}
\item Adding more advanced AI features so that the bot is able to choose
the best/most appropriate move -> Tactical-Bot

\begin{itemize}
\item Tactical-Bot will use defined tactics to choose a move which suits
the goals it's trying to achieve
\item General tactics need to be defined based on Diplomacy rules (to make
it as general as possible with the possbility of using expandable
maps)

\begin{itemize}
\item Such as prioritising keeping all your provinces in your territory/one
area
\item Or trying to capture as much opposing provinces as possible and not
care as much about your own
\item Or capturing other provinces may be more important than keeping your
own and building as many supply centres as possible
\item Being Offensive (aggresive) or Defensive (passive)
\item AI Techniques and theories used to implement such heuristics are discussed
below in {[}2.3 AI Techniques and Design{]}
\end{itemize}
\item \textasciicircum{} {*}NEED TO ADD MORE THINGS ABOUT AI SPECIFICATION{*}
\item Many possibilities here and taking the time to experiment around with
the tactics will show us which ones are more effective against others
\end{itemize}
\item In conjunction, negotation can also implemented into a bot -> Negotiation-Bot

\begin{itemize}
\item Will be able to deal with negotiating with other bot/players
\item The code base for negotiation bot will be seperate but can me made
to work together with Tactical-Bot as one bot
\item This ensures good modularity between the two bots and means that tactics
and negotiations can be built seperately to minimise interference
between the two
\end{itemize}
\end{enumerate}

\subsection{AI Techniques and Design}

{[}Matthias's bit{]}


\section{People Management}


\subsection{Team Management}
\begin{itemize}
\item Working in a team means that people have different skills and therefore
may end up contributing more than others, but this can be a positive
as the most effective way to work as a team is to use everyone's skills
effectively
\item The two main languages used are Haskell and Haxe. There is one member
who is strongest at Haskell and one who is strongest at Haxe

\begin{itemize}
\item This has meant that for the first two iterations, the two members
mentioned above have led the project and contributed the most
\item However for the next 2 iterations, contributions should be more even
as the rest of the group has had time to learn the languages required
and can make a greater contribution
\end{itemize}
\end{itemize}

\section{Ethical and Environment Impact}


\subsection{Ethical Issues}

Users who register for a game on the system may be required to leave
their email addresses and names for authentication. This means their
personal details need to be stored safely and ensure that it is stored
safely and away from any possible attacks on the server. In addition
this personal information needs to be handled with care and that it
is never given out under any circumstances.\\
In order to avoid any ethical issues we would have to ensure that
only team members have access to the database holding this information
and perhaps we would need some sort of encryption on the table itself.


\subsection{Environmental Impact }
\end{document}
